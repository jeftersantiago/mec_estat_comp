\documentclass[a4paper, 11pt]{article}
\usepackage[margin=0.5in]{geometry}
\usepackage{framed}
\usepackage{graphicx}
\usepackage{xcolor}
\usepackage{blindtext}
\usepackage{xcolor}
\usepackage{mdframed}
\usepackage{indentfirst}
\usepackage{hyperref}
\usepackage{txfonts}
\usepackage{amsmath}
\usepackage{titling}
\usepackage{titlesec}
\usepackage[brazil]{babel}
\definecolor{LightGray}{gray}{0.97}
\usepackage{minted}
\usepackage{xcolor} % to access the named colour LightGray

\setminted[fortran]{  framesep=2mm,
  baselinestretch=1.2,
  bgcolor=LightGray,
  fontsize=\footnotesize,
  linenos}

\titleformat{\section}{\normalfont\Large\bfseries}{\thesection}{1em}{}[\titlerule] 
\titleformat{\subsection}{\normalfont\large\bfseries}{\thesubsection}{1em}{} 
% \renewcommand\thesubsection{\Alph{subsection}}
\graphicspath{ {./graficos/} }

\begin{document}
\noindent
\large\textbf{Autor:} Jefter Santiago \hfill \textbf{Projeto 4 - {\color{blue}\emph{Modelos de Crescimento}}}   \\
\#USP: 12559016 \\
\normalsize Curso: Física Estatística Computacional \hfill 2024.1 \\
Prof. F. C. Alcaraz \hfill Data de entrega: \\
\noindent\rule{7in}{2.8pt}


\section*{Anotações da aula}

- bit de configuracao de uma posicao \( b_i = 0, 1 \) um bit de informacao p/ uma rede/cadeia
\( i = 1, 2, \cdots , L \). 

- configuracao do sistema \( C_t = \{ b_1^t, b_2^2, \cdots , b_L^t\} \) 

- evolucao \( C_{t+1} = f(C_t) \), \( f \rightarrow \) regra de crescimento 
- regra mais simples: \( b_i^{t + 1} = f(b_{i-1}^t, b_i^t, b_{i+1}^t)\) cada posição
depende dos adjacentes.

- Modelo Eden e DLA -> 
























\end{document}
%%% Local Variables:
%%% mode: latex
%%% TeX-master: "main.tex"
%%% End:
