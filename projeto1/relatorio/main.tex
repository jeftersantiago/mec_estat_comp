\documentclass[a4paper, 11pt]{article}
\usepackage[margin=0.5in]{geometry}
\usepackage{framed}
\usepackage{graphicx}
\usepackage{xcolor}
\usepackage{blindtext}
\usepackage{xcolor}
\usepackage{mdframed}
\usepackage{indentfirst}
\usepackage{hyperref}
\usepackage{txfonts}
\usepackage{amsmath}
\usepackage{titling}
\usepackage{titlesec}

\titleformat{\section}{\normalfont\Large\bfseries}{\thesection}{1em}{}[\titlerule] 
\titleformat{\subsection}{\normalfont\large\bfseries}{\thesubsection}{1em}{} 
\renewcommand\thesubsection{\Alph{subsection}}

\begin{document}
\noindent
\large\textbf{Autor:} Jefter Santiago \hfill \textbf{PROJETO 1 : \emph{Análise espectral por transformadas de fourier}}   \\
\#USP: 12559016 \\
\normalsize Curso: Física Estatística Computacional 
Prof. F. C. Alcaraz \hfill Data de entrega: 23/03/2024 \\
\noindent\rule{7in}{2.8pt}


\section{Transformada de Fourier discreta}

\section{Gerando séries}

\begin{equation}
  y_i = a_1 \cos (\omega_1 t_i)   + a_2 \sin (\omega_2 t_i) , t_i = i \cdot  \Delta t , i = 1, \cdots , N
  \label{eq:series}
\end{equation}

\subsection{$N = 200, \Delta t = 0.04, a_1 = 2, a_2 = 4, \omega_1 = 4\pi (Hz), \omega_2 = 2.5 \pi (Hz)$}
\subsection{$N = 200, \Delta t = 0.04, a_1 = 3, a_2 = 2, \omega_1 = 4\pi (Hz), \omega_2 = 2.5 \pi (Hz)$}
\subsection{$N = 200, \Delta t = 0.4, a_1 = 2, a_2 = 4, \omega_1 = 4\pi (Hz), \omega_2 = 0.2 \pi (Hz)$}
\subsection{$N = 200, \Delta t = 0.04, a_1 = 3, a_2 = 2, \omega_1 = 4\pi (Hz), \omega_2 = 0.2 \pi (Hz)$}


\end{document}
%%% Local Variables:
%%% mode: latex
%%% TeX-master: "main.tex"
%%% End:
