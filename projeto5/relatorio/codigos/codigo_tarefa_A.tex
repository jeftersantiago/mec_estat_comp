\begin{minted}{fortran}

    open(1, file="saidas/tarefa-1/saida-tarefa-A1-conf-L60.dat")
    open(3, file="saidas/tarefa-1/saida-tarefa-A1-conf-L100.dat")
    open(5, file="saidas/tarefa-1/saida-tarefa-A2-conf-L60.dat")
    open(7, file="saidas/tarefa-1/saida-tarefa-A2-conf-L100.dat")

    open(2, file="saidas/tarefa-1/saida-tarefa-A1-energia-L60.dat")
    open(4, file="saidas/tarefa-1/saida-tarefa-A1-energia-L100.dat")
    open(6, file="saidas/tarefa-1/saida-tarefa-A2-energia-L60.dat")
    open(8, file="saidas/tarefa-1/saida-tarefa-A2-energia-L100.dat")
    
    call tarefa1(60, 3.0, 1, 2)
    call tarefa1(60, 0.1, 5, 4)

    call tarefa1(100, 3.0, 3, 6)
    call tarefa1(100, 0.1, 7, 8)

    do i = 1, 8
        close(i)    
    end do
    end

    subroutine tarefa1(L_real, beta, fname1, fname2)
        implicit integer(f-f)
        implicit real(m-m)
        parameter(L = 100)

        dimension exps(-4:4)
        byte lattice(1:L, 1:L)
        ! periodic boundary conditions
        dimension ipbc(0:L+1)
        ! this or using mod

        N = L_real * L_real
        ! setting ipbc
        do i = 1, L_real
            ipbc(i) = i
        end do  
        ipbc(0) = L_real
        ipbc(L_real+1) = 1

        m = 0

        call define_exponentials(exps, beta)

        call initialize_lattice(lattice, L_real, L_real)

         ! initial energy
        E = H_0(lattice, ipbc, L_real)

        write(fname2, *) 0, E
        call srand(iseed)
         ! intialize monte carlo dynamics
        do k = 1, 3000
            ! sweeps all configurations
            ! randomly flips spins
            do i = 1 , N
               call flip_spin(lattice,ipbc,exps,E,m,L_real)
            end do  
            write(fname2, *) k, E / N
        end do  
        call write_lattice(lattice, L_real, fname1) 
    end subroutine tarefa1
\end{minted}