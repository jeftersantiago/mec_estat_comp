\documentclass{tufte-handout}
%\documentclass[a4paper, 12pt]{article}
\usepackage[utf8]{inputenc}

\usepackage{geometry}
\usepackage{graphicx}
\usepackage{xcolor}
\usepackage{verbatim}
\usepackage{xcolor}
\usepackage{amssymb}

\usepackage{caption}

\usepackage{indentfirst}
\usepackage{amsmath}
\usepackage{hyperref}
\usepackage{cleveref}
\usepackage[brazil]{babel}
\usepackage{minted}
\geometry{
  a4paper,
  total={170mm,350mm},
  left=10mm,
  top=5mm,
  bottom=30mm,
  textwidth=.60\paperwidth}

\iflarger
  \AtBeginDocument{%
  \fontsize{14}{16}\selectfont
  }
%\else
\fi


\definecolor{LightGray}{gray}{0.97}

\setminted[fortran]{  
    framesep=2mm,
    baselinestretch=1.2,
    bgcolor=LightGray,
    fontsize=\footnotesize,
    linenos,
} 

\graphicspath{ ./graficos/}

\hypersetup{
  pdfauthor={Jefter Santiago},
  pdftitle={Projeto 5 - Simulações de Monte Carlo},
  pdfcreator={Jefter Santiago}, 
  colorlinks=true,    % Color links instead of boxes
  linkcolor=blue,     % Color of internal links
  citecolor=green,    % Color of citation links
  urlcolor=blue,      % Color of URLs
}

% \usepackage{tgpagella}
% \usepackage{mathpazo} 

\title{Projeto 5 - Simulações de Monte Carlo}
\author{Jefter Santiago (12559016)}
\date{Entrega: 08/06/2024}

\begin{document}
\maketitle
\section{Introdução ao modelo de Ising}

Seja uma malha $2D$ com $L_x, L_y$ sítios nos eixos 
$x$ e $y$, respectivamente. Cada sítio tem um spin $\sigma_k = \{ -1, +1 \}$ associado. 

\begin{equation}
  \mathcal{H} = -\frac{J}{2} \sum_{i = 1}^{L_x}\sum_{\ell = 1}^{L_y} s(i, \ell) \left[s_{i-1,\ell}+s_{i+1, \ell}+s_{i, \ell-1}+s_{i, \ell + 1}\right]
  \label{eq:hamiltoniano_ising}
\end{equation}

trabalhamos apenas com malhas regulares quadradas, então $L \cdot L = N$, com $L=L_x=L_y$. 

a energia total é 

\begin{equation}
    E = \langle \mathcal{H} \rangle
\end{equation}


Buscamos entender a relação da energia, magnetização 
e a temperatura nesse sistema. 
Partindo da função de partição 

\begin{equation}
    \mathcal{Z}(\beta) = \sum_{i = 1}^{L} \sum_{\ell = 1}^{L} e^{-\beta E}
    \label{eq:funcao_particao}
\end{equation} 

onde $E$ é a energia dada pela (\ref{eq:hamiltoniano_ising}) e $\beta = 1/(K_bT)$.

Além disso, o calor específico pode ser obtido por 


\begin{equation}
    C = \frac{1}{N}\frac{\langle \mathcal{H}^2 \rangle - \langle \mathcal{H} \rangle^2}{k_B T^2}
    \label{eq:calor_especifico}
\end{equation}


Partindo do hamiltoniano (\ref{eq:hamiltoniano_ising}) e o spin $\sigma_k$ em cada sítio
$(i, \ell)$ podemos fazer diversas medidas como do sistema. 
Nesse trabalho estamos interessandos principalemente na energia média e magnetização média por sítio. 
Sendo a magnetização de um sítio sendo o estado do seu spin, dizemos que a magnetização 
total é simplesmente a soma de todos spins normalizada pela quantidade total de spins $N$.

\begin{equation}
    \langle m \rangle = \frac{1}{N} \sum_{i = 1}^{L} \sum_{\ell = 1}^{L} s(i,  \ell) 
    \label{eq:mag_media1}
\end{equation}

Da descrição da magnetização e a eq.(\ref{eq:mag_media1}) já podemos intuir que as configurações
correspondentes à máximo ou mínimo de magnetização são todos spins alinhados, sendo 
$\langle m \rangle_{\text{max}} = \pm 1$.

Queremos estudar a relação dessas medidas de energia e magnetização e suas relações com a 
temperatura. Por isso buscamos uma outra relação para a magnetização média por spin. Utilizando 
a função de partição(\ref{eq:funcao_particao}) para normalização dessa medida, temos $Z$ constante de normalização
e 

\begin{equation}
  \langle m \rangle =  \frac{1}{N} \frac{1}{Z} \left( \sum_{\sigma}^{N} e^{- \beta E} s_\sigma \right)
  \label{eq:mag_media_normalizada}
\end{equation}

\begin{equation}
  P(s) = \frac{e^{\beta s \Delta M}}{e^{-\beta s \Delta M} + e^{\beta s \Delta M}}
  \label{eq:prob_spin}
\end{equation}
\begin{equation}
  P(-s) = \frac{e^{-\beta s \Delta M}}{e^{-\beta s \Delta M} + e^{\beta s \Delta M}} 
 \label{eq:prob_spin_menos}
\end{equation}

onde $\Delta M$ é 

\begin{equation}
  \Delta M = J \left[ s(i-1, \ell) + s(i+1, \ell) + s(i, \ell -1) +  s(i, \ell+1) \right]
\end{equation}

Com essa descrição da magnetização média por spin podemos impor alguma dinâmica para o sistema e observar 
as medidas desejadas.
A dinâmica que trabalhamos nesse projeto que consiste em realizar alterar 
algum spin aleatório da malha de acordo com a probabilidade, isto é, sorteamos a chance de flipar o spin 
e se for menor ou maior (\ref{eq:prob_spin}) (\ref{eq:prob_spin_menos}) o estado é alterado e realizamos as medidas desejadas. 
Utilizamos simulação de Monte Carlo para realizar a dinâmica até que o sistema atinja um determinado equilíbrio. Nesse
método cada passo da simulação corresponde à dinâmica de alterar um spin para os $N$ spins do sistema. Tipicamente nas simulações
realizadas o número de passos de Monte Carlo utilizados foi $3000$, mas para observar alguns fenômenos foi necessário aumentar esse 
número. 


\section{Detalhes de implementação}

Todos os programas implementados no projeto seguem estruturas parecidas, portanto, foi
criado um módulo apenas com operações básicas envolvendo o modelo de Ising em $2$D. No arquivo
\verb|ising_modules.f| localizado na raiz do projeto estão as funções e rotinas gerais utilizadas
nas diferentes simulações.

Na descrição da malha 2D foram adotadas condições periódicas de contorno, então as bordas da grade 
estão conectadas e na discretização das posições dos spins foi utilizado um vetor de posição definido como 

\begin{minted}{fortran}
    dimension ipbc(0:L+1)
    N = L * L
    ! setting periodic boundary conditions
    do i = 1, L
        ipbc(i) = i
    end do  
    ipbc(0) = L
    ipbc(L+1) = 1
\end{minted}

Além disso, como as (\ref{eq:prob_spin}) (\ref{eq:prob_spin_menos}) envolvem exponenciais
e são contas feitas várias vezes no programa, é mais eficiente armazenar o valor das exponenciais em vetores 
para cada um dos vizinhos $(i+1, \ell), (i-1, \ell), (i, \ell+1), (i, \ell -1)$  de um spin em $(i, \ell)$.

Algumas escolhas nas implementações podem ter sido ineficientes, como ter uma rotina apenas para initializar o calculo 
da magnetização. Embora isso seja verdade, optei por fazer dessa forma para obter um código mais 
limpo e fácil de trabalhar. Entretando poderia ter feito esse tipo de conta na inicialização da configuração 
da grade.

Segue abaixo as rotinas gerais utilizadas para medidas que envolvem o modelo de Ising:
\input{codigos/codigo_ising_modules.tex}

\section{Tarefa A - \emph{Dinâmica de Monte Carlo para temperaturas fixas}}

\begin{figure}[h!]
    \centering
    \includegraphics[width=0.4\linewidth]{tarefa-A/posicoes-iniciais.png}
    \caption{Posições iniciais das partículas.}
    \label{fig:posicoes-iniciais-a}
\end{figure}

\begin{figure}[h!]
    \centering 
    \includegraphics[width=0.4\linewidth]{tarefa-A/posicoes-finais.png}
    \label{fig:posicoes-finais-a}
    \caption{Coordenadas das partículas projetadas à cada $3 \Delta t$.}
\end{figure}

\begin{figure}[h!]
    \centering 
    \includegraphics[width=0.4\linewidth]{tarefa-A/energia.png}
    \caption{Energia do sistema à cada $3 \Delta t$.}
    \label{fig:energia_a}
\end{figure}

\clearpage
\subsection*{Código}
O código abaixo está no diretório \verb|tarefa-a/| e contém 
as simulações referentes às tarefas A, B e parte da D.
\input{codigos/tarefa-a.tex}
\clearpage
%\clearpage

\section{Tarefa B - \emph{Processos de recozimento e têmpera}}
\subsection{B.1 - \emph{Recozimento}}
O programa desenvolvido para essa simulação está abaixo: 

\begin{marginfigure}
    \includegraphics[width=0.8\linewidth]{graficos/tarefa-2/graf-tarefa-B1-conf-inicial.png}
    \caption{Configuração inicial da simulação. $\beta = 1/2$ }
    \label{fig:b1_conf_inicial}
\end{marginfigure}

\input{codigos/codigo_tarefa_B1.tex}

Fazendo evolução da temperatura de forma lenta, com $\Delta \beta = 0.001$ temos o processo de recozimento. 
Partimo do sistema desordenado, com temperatura infinita e a cada passo de Monte Carlo provocamos uma variação
de temperatura $\Delta \beta$. A figura (\ref{fig:b1_conf_inicial}) mostra a configuração inicial do sistema de spins.



Estamos interessados em observar a energia média por spin. Pelo gráfico 
abaixo(\ref{fig:tarefa_b1_graf_energia}) nota-se que a energia média parte de zero, pois o sistema 
está completamente desordenado, e decresce até atingir  a energia limite em $-2$. 

\begin{figure}
    \centering
    \includegraphics[width=0.6\linewidth]{graficos/tarefa-2/graf-tarefa-B1-mag-eng.png}
    \caption{Energia média de spin por iterações de Monte Carlo.}
    \label{fig:tarefa_b1_graf_energia}
\end{figure}


Além disso, temos a configuração final dos spins do sistema bidimensional(\ref{fig:b1_conf_final}). 
Há uma faixa de magnetização na malha, a presença dela deve estar associada ao número de iterações de Monte Carlo feita
utilizado na simulação ($3000$ passos) que não foi o bastante para o sistema ficar todo alinhado. 

\begin{marginfigure}
    \centering
    \includegraphics[width=0.8\linewidth]{graficos/tarefa-2/graf-tarefa-B1-conf-final.png}
    \caption{Configuração final da malha 2D após dinâmica de recozimento.}
    \label{fig:b1_conf_final}
\end{marginfigure}

% \clearpage
\subsection{B.2 - \emph{Tempera}}
O código dessa simulação é quase idêntico ao da tarefa anterior e está compilado abaixo: 
\input{codigos/codigo_tarefa_B2.tex}

Nessa simulação partimos da mesma configuração inicial que a anterior 
e  variamos o $\beta$ de maneira brusca e o sistema pode não atingir o equiblirio. Foi utilizada 
uma malha de tamanho $L= 90$ para essa simulação e mesmo número de passos. 

\begin{figure}
    \centering
    \includegraphics[width=0.6\linewidth]{graficos/tarefa-2/graf-tarefa-B2-mag-eng.png}
    \caption{Energia média de spin por iterações de Monte Carlo.}
\end{figure}

Nota-se que a energia média decaí muito mais rapidamente que no caso anterior(\ref{fig:tarefa_b1_graf_energia})
e a configuração final consegue atingir o equiblirio(\ref{fig:b2_conf_final}). Diferentemente do processo anterior, na têmpera 
os spins vizinhos conseguem se alinhar no tempo de Monte Carlo utilizado na simulação.

\begin{marginfigure}
    \centering
    \includegraphics[width=0.8\linewidth]{graficos/tarefa-2/graf-tarefa-B2-conf-final.png}
    \caption{Configuração final para rede de spins na dinâmica de têmpera.}
    \label{fig:b2_conf_final}
\end{marginfigure}
%\clearpage

\section{Tarefa C - \emph{Loop térmico}}
\subsection{C.1 - Histerese }
Segue abaixo a implementação da simulção de histerese:
\input{codigos/codigo_tarefa_C1.tex}

No gráfico (\ref{fig:c1_dbeta1}) temos o comportamento da energia média por spin na dinâmica do loop
térmico e o gráfico de histerese, isto é, a energia média em relação à $\beta$ para 
variações de $\Delta b = 0,001$. 

\begin{figure}
    \centering
    \includegraphics[width=0.8\linewidth]{graficos/tarefa-3/graf-tarefa-C1-delta1.png}
    \caption{À esquerda energia média por spin por iterações de Monte Carlo e à direita em relação à $\beta$.}
    \label{fig:c1_dbeta1}
\end{figure}


A figura (\ref{fig:c1_dbeta2}) equivale a dinâmica como a anterior, mas com uma variação 
$\Delta \beta = 0,0001$, que fornece um resultado com menos flutuações, sobretudo para as redes maiores.

\begin{figure}
    \centering
    \includegraphics[width=0.8\linewidth]{graficos/tarefa-3/graf-tarefa-C1-delta2.png}
    \caption{À esquerda energia média por spin por iterações de Monte Carlo e à direita em relação à $\beta$.}
    \label{fig:c1_dbeta2}
\end{figure}

Podemos observar nos gráficos que a região de histerese correspondem à um intervalo de $\beta$ entre $0,4$ e 
$0,6$, mas apenas a partir dessas medidas não conseguimos ter uma boa precisão dessa medida.

\subsection{C.2 - Temperatura crítica }


\begin{marginfigure}
    \centering
    \includegraphics[width=\linewidth]{graficos/tarefa-3/graf-tarefa-C2-conf.png}
    \caption{Configuração inicial para dinâmica utilizada na medida de temperatura crítica.}
    \label{fig:c2_conf_inicial}
\end{marginfigure}


Modificação no código do item anterior: 

\begin{minted}{fortran}

        dimension betas(1:5)
        parameter(betas = (/0.41, 0.44, 0.47, 0.51, 0.55/))

        open(1, file="saidas/tarefa-3/saida-tarefa-C2-L60-b1.dat")
        open(2, file="saidas/tarefa-3/saida-tarefa-C2-L60-b2.dat")
        open(3, file="saidas/tarefa-3/saida-tarefa-C2-L60-b3.dat")
        open(4, file="saidas/tarefa-3/saida-tarefa-C2-L60-b4.dat")
        open(5, file="saidas/tarefa-3/saida-tarefa-C2-L60-b5.dat")

        do i = 1, 5
            call tarefaC2(60, betas(i), i)
            close(1)
        end do

        open(1, file="saidas/tarefa-3/saida-tarefa-C2-L80-b1.dat")
        open(2, file="saidas/tarefa-3/saida-tarefa-C2-L80-b2.dat")
        open(3, file="saidas/tarefa-3/saida-tarefa-C2-L80-b3.dat")
        open(4, file="saidas/tarefa-3/saida-tarefa-C2-L80-b4.dat")
        open(5, file="saidas/tarefa-3/saida-tarefa-C2-L80-b5.dat")

        do i = 1, 5
            call tarefaC2(80, betas(i), i)
            close(1)
        end do

        open(1, file="saidas/tarefa-3/saida-tarefa-C2-L100-b1.dat")
        open(2, file="saidas/tarefa-3/saida-tarefa-C2-L100-b2.dat")
        open(3, file="saidas/tarefa-3/saida-tarefa-C2-L100-b3.dat")
        open(4, file="saidas/tarefa-3/saida-tarefa-C2-L100-b4.dat")
        open(5, file="saidas/tarefa-3/saida-tarefa-C2-L100-b5.dat")

        do i = 1, 5
            call tarefaC2(100, betas(i), i)
            close(1)
        end do
        end
        subroutine tarefaC2(L_real, beta, fname)
!               Tarefa B - Recozimento e quenching
            implicit integer(f-f)
            implicit real(j-j, m-m)
            parameter(L = 100)
            dimension exps(-4:4)
            byte lattice(1:L, 1:L)
            ! periodic boundary conditions
            dimension ipbc(0:L+1)

            do i = 1, L_real
                ipbc(i) = i
            end do  

            ipbc(0) = L_real
            ipbc(L_real+1) = 1

            N = L_real * L_real

            mag = 0.0d0

            call srand(L_real * 392)

            ! half ordered / half random.
            call initialize_lattice(lattice, L_real, L_real)
            call initialize_random_lattice(lattice,  L_real/2, L_real)

            open(99, file = "saidas/tarefa-3/saida-tarefa-C2-conf.dat")
            call write_lattice(lattice, L_real, 99)
            close(99)

            call total_magnetization(lattice, mag, L_real)

            ! initial energy
            E = H_0(lattice, ipbc, L_real)
            dbeta = 0.01
            write(fname, *) 0, E/N
            do i = 1, 3000
                call define_exponentials(exps, beta)
                do k = 1 , N
                    call flip_spin(lattice,ipbc,exps,E,mag,L_real)
                end do   
                write(fname, *) i, E/N
            end do
        end subroutine tarefaC2
\end{minted}


\begin{marginfigure}
    \centering
    \includegraphics[width=\linewidth]{graficos/tarefa-3/graf-tarefa-C2-L80.png}
    \caption{Dinâmica para L=80.}
    \label{fig:c2_l80}
\end{marginfigure}

Partimos dos resultados do item anterior e tentamos obter a temperatura crítica do modelo. Para isso observamos
a variação de energia no intervalo $\beta$ discutido antes, isto é, $0,4 < \beta < 0.6$. 
A imagem (\ref{fig:c2_conf_inicial}) mostra a configuração inicial do sistema. Foram escolhidos alguns 
valores de $\beta$ para executar a dinâmica de Monte Carlo. 


\begin{marginfigure}
    \centering
    \includegraphics[width=\linewidth]{graficos/tarefa-3/graf-tarefa-C2-L60.png}
    \caption{Dinâmica para L=60.}
    \label{fig:c2_l60}
\end{marginfigure}



Nas figuras (\ref{fig:c2_l60}), (\ref{fig:c2_l80}) e (\ref{fig:c2_l100}) estão 
as evoluções, em um intervalo de passos de Monte Carlo reduzido, da energia média por spin. 

\clearpage
Nota-se que as energias médias por spin sempre partem do mesmo valor no intervalo da histerese e a 
que possui maior variação é a que corresponde à $\beta = 0.44$, esse é o $\beta$ relacionado à temperatura crítica $T_c = 1/\beta_c \approx 2,27$ .

\begin{figure}
    \centering
    \includegraphics[width=0.5\linewidth]{graficos/tarefa-3/graf-tarefa-C2-L100.png}
    \caption{Dinâmica para L=100.}
    \label{fig:c2_l100}
\end{figure}

Além disso, pela (\ref{eq:calor_especifico}) podemos constatar que esse $\beta_c = 0,44$ também está associado à 
um valor específico crítico no modelo. 
%\clearpage

\section{Tarefa D - \emph{Quebra espontânea de simetria}}

\begin{equation}
    K_B T = \langle \frac{m}{2} \left( v_x^2 + v_y^2 \right) \rangle
\end{equation}
\end{document}