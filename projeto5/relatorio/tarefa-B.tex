\subsection{B.1 - \emph{Recozimento}}
O programa desenvolvido para essa simulação está abaixo: 

\begin{marginfigure}
    \includegraphics[width=0.8\linewidth]{graficos/tarefa-2/graf-tarefa-B1-conf-inicial.png}
    \caption{Configuração inicial da simulação. $\beta = 1/2$ }
    \label{fig:b1_conf_inicial}
\end{marginfigure}

\input{codigos/codigo_tarefa_B1.tex}

Fazendo evolução da temperatura de forma lenta, com $\Delta \beta = 0.001$ temos o processo de recozimento. 
Partimo do sistema desordenado, com temperatura infinita e a cada passo de Monte Carlo provocamos uma variação
de temperatura $\Delta \beta$. A figura (\ref{fig:b1_conf_inicial}) mostra a configuração inicial do sistema de spins.



Estamos interessados em observar a energia média por spin. Pelo gráfico 
abaixo(\ref{fig:tarefa_b1_graf_energia}) nota-se que a energia média parte de zero, pois o sistema 
está completamente desordenado, e decresce até atingir  a energia limite em $-2$. 

\begin{figure}
    \centering
    \includegraphics[width=0.6\linewidth]{graficos/tarefa-2/graf-tarefa-B1-mag-eng.png}
    \caption{Energia média de spin por iterações de Monte Carlo.}
    \label{fig:tarefa_b1_graf_energia}
\end{figure}


Além disso, temos a configuração final dos spins do sistema bidimensional(\ref{fig:b1_conf_final}). 
Há uma faixa de magnetização na malha, a presença dela deve estar associada ao número de iterações de Monte Carlo feita
utilizado na simulação ($3000$ passos) que não foi o bastante para o sistema ficar todo alinhado. 

\begin{marginfigure}
    \centering
    \includegraphics[width=0.8\linewidth]{graficos/tarefa-2/graf-tarefa-B1-conf-final.png}
    \caption{Configuração final da malha 2D após dinâmica de recozimento.}
    \label{fig:b1_conf_final}
\end{marginfigure}

% \clearpage
\subsection{B.2 - \emph{Tempera}}
O código dessa simulação é quase idêntico ao da tarefa anterior e está compilado abaixo: 
\input{codigos/codigo_tarefa_B2.tex}

Nessa simulação partimos da mesma configuração inicial que a anterior 
e  variamos o $\beta$ de maneira brusca e o sistema pode não atingir o equiblirio. Foi utilizada 
uma malha de tamanho $L= 90$ para essa simulação e mesmo número de passos. 

\begin{figure}
    \centering
    \includegraphics[width=0.6\linewidth]{graficos/tarefa-2/graf-tarefa-B2-mag-eng.png}
    \caption{Energia média de spin por iterações de Monte Carlo.}
\end{figure}

Nota-se que a energia média decaí muito mais rapidamente que no caso anterior(\ref{fig:tarefa_b1_graf_energia})
e a configuração final consegue atingir o equiblirio(\ref{fig:b2_conf_final}). Diferentemente do processo anterior, na têmpera 
os spins vizinhos conseguem se alinhar no tempo de Monte Carlo utilizado na simulação.

\begin{marginfigure}
    \centering
    \includegraphics[width=0.8\linewidth]{graficos/tarefa-2/graf-tarefa-B2-conf-final.png}
    \caption{Configuração final para rede de spins na dinâmica de têmpera.}
    \label{fig:b2_conf_final}
\end{marginfigure}