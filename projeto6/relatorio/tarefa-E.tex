Agora que estudamos a dinâmica molecular e conseguimos atingir observar o que as velocidades 
seguem distribuições de Maxwell-Boltzman, queremos impor uma dinâmica específica sobre o 
gás 2D. 
Nesse caso queremos observar a cristalização de moleculas, e para isso consideramos uma caixa 
de tamanho $L=4$ com $N = 16$ partículas, ou seja, a densidade nesse caso é $\sigma = 1$, o passo 
foi $\Delta t = 0.005$ e a velocidade inicial de teste $v_0 = 1$. 
No entanto, para essa velocidade não foi possível observar a cristalização acontecendo nos diferentes regimes
de temo desejados, por isso foi necessário diminuir a velocidade inicial $v_0$ para 
$v_0 = 0.2$. Isso condiz com um sistema à baixa temperaturas de alta densidade. 

Como pode ver abaixo (\ref{fig:posicoes-finais-e}) correspondem à cristalização das moléculas. 
Da esquerda para a direita temos o ``rastro'' das partículas nos temos $0 \leq t \leq 0.1$, $0.2 \leq t \leq 0.4$ e 
por fim $13 \leq t \leq 16$. Foram consideradas as posições em intervalos de $ 10 \Delta t$. 

\begin{figure}[h!]
    \centering
    \includegraphics[width=0.95\linewidth]{tarefa-E/posicoes-finais.png}
    \caption{Etapas da cristalização das moléculas.}
    \label{fig:posicoes-finais-e}
\end{figure}


Além desse gráfico foi construido um vídeo para essa tarefa. Esse é um pouco mais longo que os anteriores\footnote{E espero que consiga subir ele para o basalto.}
mas observando apenas os segundos finais dele podemos notar que as partículas permanecem quase presas em uma determinada região nas proximidades da 
do ponto de cristalização. É possível até perceber a estrutura triangular com a qual as partículas se movimentam em torno dos vizinhos. 

\subsection*{Implementação - Simulação E}
\input{codigos/tarefa-e.tex}
\clearpage 