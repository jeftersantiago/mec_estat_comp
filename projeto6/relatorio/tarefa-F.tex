Por fim, queremos elaborar um pouco mais a dinâmica da tópico anterior(\ref{sec:secE}). 
Agora iremos impor uma fusão à um sólido. Para isso consideraremos um sistema de mesmas dimensões que o anterior 
e executaremos a dinâmica normalmente até atingir a cristalização como antes. A partir desse ponto
nosso impomos uma dinâmica externa às moléculas aumentando sua velocidade por um fator $\gamma = 1.5$. Repetimos 
essa dinâmica até que o sistema atinja um novo equilíbrio térmico. 

A imposição de novas velocidades é dada por 

\begin{equation}
    \mathbf{r}_{\text{anterior}} \longleftarrow \mathbf{r}_{\text{atual}} - (\mathbf{r}_{\text{atual}}-\mathbf{r}_{\text{anterior}})\gamma
    \label{eq:update_vels}
\end{equation}

Após muitas iterações e aplicando a (\ref{eq:update_vels}) às velocidades conseguimos observar a fusão acontecendo pela 
figura(\ref{fig:posicoes-finais-f}). Nota-se que talvez a escolha de intervalo para a figura do meio talvez não tenha sido 
tão boa, mas espero que seja possível perceber o inicio da difusão das moléculas em um entorno da cristalização anterior. 

\begin{figure}[h!]
    \centering
    \includegraphics[width=0.95\linewidth]{tarefa-F/posicoes-finais.png}
    \caption{}
    \label{fig:posicoes-finais-f}
\end{figure}


E para finalizar, também foi feito um gráfico da energia desse sistema. Era de se esperar que fosse haver um 
aumentando de energia simplesmente pelo fato de ser um processo de fusão e o gráfico abaixo (\ref{fig:energia-f})
expressa esse fato assim como mostra o quão abrupto é o aumento dessa energia, já que a nossa dinâmica 
de aumento velocidades é instantâneo.

\begin{figure}[h!]
    \centering
    \includegraphics[width=0.4\linewidth]{tarefa-F/energia.png}
    \caption{Energia total a cada tempo.}
    \label{fig:energia-f}
\end{figure}


\clearpage
\subsection{Implementação - Simulação F}
Um detalhe que não consegui explicar sobre essa simulação é que a única forma que consegui fazer a fusão acontecer 
corretamente foi considerando apenas uma das componenentes na (\ref{eq:update_vels}) e não todo vetor. 

\input{codigos/tarefa-f.tex}