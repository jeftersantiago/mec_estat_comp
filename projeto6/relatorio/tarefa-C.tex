
\begin{figure}[h!]
    \centering
    \includegraphics[width=0.7\linewidth]{tarefa-C/distribuicoes-c.png}
    \caption{Distribuição da velocidade, magnitude e componentes em intervalos $t=20-40$, $t=40-60$ e $t=60-80$.}
    \label{fig:distribuicoes-velocidade-c}
\end{figure}



\subsection*{Código}
Nesse código temos a simulação desse item, considerandos as mudanças do perfil inicial de velocidades 
das partículas. E além disso ela também gera os dados necessários para o cálculo da tarefa posterior, D, assim 
como parte do código da tarefa A também fazia o mesmo.
\input{codigos/tarefa-c.tex}
\clearpage