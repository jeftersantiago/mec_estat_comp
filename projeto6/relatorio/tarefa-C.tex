Como foi discutido em (\ref{sec:sec_b}) o nosso interesse agora é realizar a mesma simulação numérica 
para distribuiçẽos de velocidades, porém para um sistema com configurações iniciais de velocidades 
diferente do anterior. 

Para isso foi necessário realizar modificações na implementação da simulação e não foi utilizado o mesmo 
programa que o da tarefa A (\ref{ssec:codigoA}). O código (\ref{ssec:codigoC}) inicializa cada métade 
das partículas com velocidade $1$ em apenas uma componente $x$ ou $y$. 
O resultado da dinâmica pode ser observado no \verb|gif| gerado e armazenado na  \verb|graficos/tarefa-C/evolucao.gif|. E 
os resultados das distribuições seguem abaixo.

\begin{figure}[h!]
    \centering
    \includegraphics[width=0.7\linewidth]{tarefa-C/distribuicoes-c.png}
    \caption{Distribuição da velocidade, magnitude e componentes em intervalos $t=20-40$, $t=40-60$ e $t=60-80$.}
    \label{fig:distribuicoes-velocidade-c}
\end{figure}


De imediato observa-se que toda distribuições parecem seguir as curvas teóricas, no entanto, quase todas 
elas são transladadas. A distribuição das magnitudes só se aproxima muito bem da curva teórica para um tempo muito avançado $t=60-80$, ponto em que o sistema deve estar 
próximo do equilíbrio térmico.


Nota-se que esse resultado nos mostra que podemos atingir o equilíbrio independente do perfil inicial de 
velocidades empregado às moléculas do sistema. Mas, o ``tradeoff'' que precisamos realizar nesse caso é 
executar a dinâmica por muito mais iterações. 

\subsection{Implementação - Simuação C}
\label{ssec:codigoC}
Nesse código temos a simulação desse item, considerandos as mudanças do perfil inicial de velocidades 
das partículas. E além disso ela também gera os dados necessários para o cálculo da tarefa posterior, D, assim 
como parte do código da tarefa A também fazia o mesmo.
\input{codigos/tarefa-c.tex}
\clearpage