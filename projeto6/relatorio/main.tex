\documentclass[a4paper, 13pt]{article}
\usepackage[margin=0.7in]{geometry}
\usepackage{framed}
\usepackage{graphicx}
\usepackage{xcolor}
\usepackage{blindtext}
\usepackage{xcolor}
\usepackage{mdframed}
\usepackage{indentfirst}
\usepackage{graphicx}
\usepackage{subcaption}
\usepackage{hyperref}
\usepackage{verbatim}
\usepackage{amsmath}
\usepackage{titling}
\usepackage{titlesec}
\usepackage[brazil]{babel}
\definecolor{LightGray}{gray}{0.97}
\usepackage{minted}
\usepackage{xcolor}


\usepackage{biblatex}
\addbibresource{refs.tex}


\setminted[fortran]{
  framesep=2mm,
  baselinestretch=1.2,
  bgcolor=LightGray,
  fontsize=\footnotesize,
  linenos}
  \graphicspath{{../src/graficos/}}
\hypersetup{
  pdfauthor={Jefter Santiago},
  pdftitle={Projeto 6 - Dinâmica Molecular},
  pdfcreator={Jefter Santiago}, 
  pdflang={Portuguese},
  colorlinks=true,    % Color links instead of boxes
  linkcolor=blue,     % Color of internal links
  citecolor=green,    % Color of citation links
  urlcolor=blue,      % Color of URLs
}
\title{\color{blue}Projeto 6 - Dinâmica molecular}
\author{Jefter Santiago (12559016)}
%\date{Entrega: 01/06/2024}
\begin{document}
\maketitle
\section{Introdução}
Potencial de Lennard-Jonnes

\begin{equation}
    \mathcal{U}(\vec{r}) = 4 \epsilon \left[ \left(\frac{\sigma}{r} \right)^12 - \left(\frac{\sigma}{r}\right)^6 \right]
    \label{eq:potencial_lanner_jones}
\end{equation}

Utilizando unidades genericas tal que $m=1$, então, pela segunda lei de Newton temos que as componentes 
da velocidade são  

\begin{align}
    a^{x}_{j} = \sum_{j = 1}^{N} \sum_{\ell \neq j}^{N} F_{j, \ell} \cos(\theta_{j, \ell})
    a^{y}_{j} = \sum_{j = 1}^{N} \sum_{\ell \neq j}^{N} F_{j, \ell} \sin(\theta_{j, \ell})
\end{align}

porém é mais interessante encontrar uma relação que não envolva as funções trigonométricas. De fato, 
podemos escrever os senos e cossenos de $\theta_{j, \ell}$ 


$$ \sin(\theta_{j, \ell}) = \frac{dy_{j, \ell}}{d_{j, \ell}} $$
$$ \cos(\theta_{j, \ell}) = \frac{dx_{j, \ell}}{d_{j, \ell}} $$


\subsubsection{Algoritmo de Verlet}

\begin{align}
    t &= n \Delta t \\ 
    x_i(n+1) &= 2 x_i (n) - x_i (n-1)+ a_i^x (n) (\Delta t)^2, n \ge 1 \\ 
    y_i(n+1) &= 2 y_i (n) - y_i (n-1)+ a_i^y (n) (\Delta t)^2, n \ge 1 \\  
    x_i(1) &= x(0) + v_i^x (0) \Delta t \\ 
    y_i(1) &= y(0) + v_i^y (0) \Delta t \\ 
    v_i^x (n) &= \left( x_i(n+1) - x_i(n-1) \right) /(2 \Delta t)\\
    v_i^y (n) &= \left(  y_i(n+1) - y_i(n-1) \right) /(2 \Delta t)
\end{align}
\subsection{Detalhes de implementação}

Módulo para simulações de dinâmica molecular
\begin{minted}{fortran}
    ! Submodules for molecular dynamic simulations
    
    ! Velocity delta 
    function delta_pbc(r_next, r_prev,L)
          implicit real*8(a-h, o-y)
          delta_pbc = r_next - r_prev
          delta_pbc = delta_pbc - L * nint(delta_pbc / L)
    end function delta_pbc

    subroutine initialize_particles(N, L, r_curr,r_prev, v, v0)
          implicit real*8(a-h, o-y)
          dimension r_prev(20, 2)
          dimension r_curr(20, 2)
          dimension v(20, 2)
         
          ! Defining # rows/columns 
          n_cols = ceiling(sqrt(N*1d0))
          n_rows = ceiling((N*1d0)/(n_cols*1d0)) 
          
          ! Spacing 1/4 
          x_spacing = L/(1d0*n_cols)
          y_spacing = L/(1d0*n_rows)
          spacing = min(x_spacing, y_spacing)/4.0 
          
          ! Centering in the grid
          x_offset = x_spacing / 2.0 
          y_offset = y_spacing / 2.0
          call srand(562369)

          k = 1 
          do j = 1, n_rows 
                do i = 1, n_cols 
                      r_curr(k, 1) = (i-1)*x_spacing+x_offset
                      r_curr(k, 2) = (j-1)*y_spacing+y_offset
                      
                      r_curr(k, 1) = r_curr(k,1)+(rand())*spacing
                      r_curr(k, 2) = r_curr(k,2)+(rand())*spacing

                      theta = 2*pi*rand()
                      v(k, 1) = v0*cos(theta)
                      v(k, 2) = v0*sin(theta)
                      
                      r_prev(k, 1) = r_curr(k, 1) - v(k, 1) * dt 
                      r_prev(k, 2) = r_curr(k, 2) - v(k, 2) * dt 
                      k=k+1
                end do 
          end do
    end subroutine initialize_particles

    ! Updates acceleration a = ax, ay 
    ! between particle i and all others
    subroutine compute_acc(N,i,j,L,r_curr,acc, r)
          implicit real*8(a-h, o-y)
          dimension r_curr(20, 2)
          dimension acc(2)
          dimension r(20, 20)
          epsilon = 1e-3

          dx = r_curr(i, 1) - r_curr(j, 1)
          dy = r_curr(i, 2) - r_curr(j, 2)

          dx = dx - L * nint(dx / L)
          dy = dy - L * nint(dy / L)

          r_ij = sqrt(dx**2 + dy**2)
          
          r(i, j) = r_ij 
          r(j, i) = r_ij

          if(r_ij > epsilon .and. r_ij <= 3d0) then 
                F = 24.0 * (2d0/r_ij**13 - 1d0/r_ij**7)
                acc(1) = acc(1) + F * dx / r_ij 
                acc(2) = acc(2) + F * dy / r_ij
          end if 
    end subroutine compute_acc

    subroutine compute_energy(N, L, v, r_curr, E, r)
          implicit real*8(a-h, o-y)
          dimension v(20, 2)
          dimension r_curr(20, 2)
          dimension r(20, 20)
          
          epsilon = 1e-3
          Tk = 0d0
          do i = 1, N
              Tk = Tk + 0.5 * (v(i, 1)**2 + v(i, 2)**2)
          end do
          U = 0d0
          do i = 1, N
            do j = i + 1, N
                r_ij = r(i, j)

                if (r_ij > epsilon .and. r_ij <= 3d0) then
                    U = U + 4 * (r_ij**(-12) - r_ij**(-6))
                end if
            end do
          end do
          E = Tk + U
    end subroutine compute_energy
\end{minted}
\clearpage
\section{Tarefa A}

\begin{figure}[h!]
    \centering
    \includegraphics[width=0.4\linewidth]{tarefa-A/posicoes-iniciais.png}
    \caption{Posições iniciais das partículas.}
    \label{fig:posicoes-iniciais-a}
\end{figure}

\begin{figure}[h!]
    \centering 
    \includegraphics[width=0.4\linewidth]{tarefa-A/posicoes-finais.png}
    \label{fig:posicoes-finais-a}
    \caption{Coordenadas das partículas projetadas à cada $3 \Delta t$.}
\end{figure}

\begin{figure}[h!]
    \centering 
    \includegraphics[width=0.4\linewidth]{tarefa-A/energia.png}
    \caption{Energia do sistema à cada $3 \Delta t$.}
    \label{fig:energia_a}
\end{figure}

\clearpage
\subsection*{Código}
O código abaixo está no diretório \verb|tarefa-a/| e contém 
as simulações referentes às tarefas A, B e parte da D.
\input{codigos/tarefa-a.tex}
\clearpage
\section{Tarefa B}
\label{sec:sec_b}
\subsection{B.1 - \emph{Recozimento}}
O programa desenvolvido para essa simulação está abaixo: 

\begin{marginfigure}
    \includegraphics[width=0.8\linewidth]{graficos/tarefa-2/graf-tarefa-B1-conf-inicial.png}
    \caption{Configuração inicial da simulação. $\beta = 1/2$ }
    \label{fig:b1_conf_inicial}
\end{marginfigure}

\input{codigos/codigo_tarefa_B1.tex}

Fazendo evolução da temperatura de forma lenta, com $\Delta \beta = 0.001$ temos o processo de recozimento. 
Partimo do sistema desordenado, com temperatura infinita e a cada passo de Monte Carlo provocamos uma variação
de temperatura $\Delta \beta$. A figura (\ref{fig:b1_conf_inicial}) mostra a configuração inicial do sistema de spins.



Estamos interessados em observar a energia média por spin. Pelo gráfico 
abaixo(\ref{fig:tarefa_b1_graf_energia}) nota-se que a energia média parte de zero, pois o sistema 
está completamente desordenado, e decresce até atingir  a energia limite em $-2$. 

\begin{figure}
    \centering
    \includegraphics[width=0.6\linewidth]{graficos/tarefa-2/graf-tarefa-B1-mag-eng.png}
    \caption{Energia média de spin por iterações de Monte Carlo.}
    \label{fig:tarefa_b1_graf_energia}
\end{figure}


Além disso, temos a configuração final dos spins do sistema bidimensional(\ref{fig:b1_conf_final}). 
Há uma faixa de magnetização na malha, a presença dela deve estar associada ao número de iterações de Monte Carlo feita
utilizado na simulação ($3000$ passos) que não foi o bastante para o sistema ficar todo alinhado. 

\begin{marginfigure}
    \centering
    \includegraphics[width=0.8\linewidth]{graficos/tarefa-2/graf-tarefa-B1-conf-final.png}
    \caption{Configuração final da malha 2D após dinâmica de recozimento.}
    \label{fig:b1_conf_final}
\end{marginfigure}

% \clearpage
\subsection{B.2 - \emph{Tempera}}
O código dessa simulação é quase idêntico ao da tarefa anterior e está compilado abaixo: 
\input{codigos/codigo_tarefa_B2.tex}

Nessa simulação partimos da mesma configuração inicial que a anterior 
e  variamos o $\beta$ de maneira brusca e o sistema pode não atingir o equiblirio. Foi utilizada 
uma malha de tamanho $L= 90$ para essa simulação e mesmo número de passos. 

\begin{figure}
    \centering
    \includegraphics[width=0.6\linewidth]{graficos/tarefa-2/graf-tarefa-B2-mag-eng.png}
    \caption{Energia média de spin por iterações de Monte Carlo.}
\end{figure}

Nota-se que a energia média decaí muito mais rapidamente que no caso anterior(\ref{fig:tarefa_b1_graf_energia})
e a configuração final consegue atingir o equiblirio(\ref{fig:b2_conf_final}). Diferentemente do processo anterior, na têmpera 
os spins vizinhos conseguem se alinhar no tempo de Monte Carlo utilizado na simulação.

\begin{marginfigure}
    \centering
    \includegraphics[width=0.8\linewidth]{graficos/tarefa-2/graf-tarefa-B2-conf-final.png}
    \caption{Configuração final para rede de spins na dinâmica de têmpera.}
    \label{fig:b2_conf_final}
\end{marginfigure}
\section{Tarefa C}
\label{sec:sec_C}
\subsection{C.1 - Histerese }
Segue abaixo a implementação da simulção de histerese:
\input{codigos/codigo_tarefa_C1.tex}

No gráfico (\ref{fig:c1_dbeta1}) temos o comportamento da energia média por spin na dinâmica do loop
térmico e o gráfico de histerese, isto é, a energia média em relação à $\beta$ para 
variações de $\Delta b = 0,001$. 

\begin{figure}
    \centering
    \includegraphics[width=0.8\linewidth]{graficos/tarefa-3/graf-tarefa-C1-delta1.png}
    \caption{À esquerda energia média por spin por iterações de Monte Carlo e à direita em relação à $\beta$.}
    \label{fig:c1_dbeta1}
\end{figure}


A figura (\ref{fig:c1_dbeta2}) equivale a dinâmica como a anterior, mas com uma variação 
$\Delta \beta = 0,0001$, que fornece um resultado com menos flutuações, sobretudo para as redes maiores.

\begin{figure}
    \centering
    \includegraphics[width=0.8\linewidth]{graficos/tarefa-3/graf-tarefa-C1-delta2.png}
    \caption{À esquerda energia média por spin por iterações de Monte Carlo e à direita em relação à $\beta$.}
    \label{fig:c1_dbeta2}
\end{figure}

Podemos observar nos gráficos que a região de histerese correspondem à um intervalo de $\beta$ entre $0,4$ e 
$0,6$, mas apenas a partir dessas medidas não conseguimos ter uma boa precisão dessa medida.

\subsection{C.2 - Temperatura crítica }


\begin{marginfigure}
    \centering
    \includegraphics[width=\linewidth]{graficos/tarefa-3/graf-tarefa-C2-conf.png}
    \caption{Configuração inicial para dinâmica utilizada na medida de temperatura crítica.}
    \label{fig:c2_conf_inicial}
\end{marginfigure}


Modificação no código do item anterior: 

\begin{minted}{fortran}

        dimension betas(1:5)
        parameter(betas = (/0.41, 0.44, 0.47, 0.51, 0.55/))

        open(1, file="saidas/tarefa-3/saida-tarefa-C2-L60-b1.dat")
        open(2, file="saidas/tarefa-3/saida-tarefa-C2-L60-b2.dat")
        open(3, file="saidas/tarefa-3/saida-tarefa-C2-L60-b3.dat")
        open(4, file="saidas/tarefa-3/saida-tarefa-C2-L60-b4.dat")
        open(5, file="saidas/tarefa-3/saida-tarefa-C2-L60-b5.dat")

        do i = 1, 5
            call tarefaC2(60, betas(i), i)
            close(1)
        end do

        open(1, file="saidas/tarefa-3/saida-tarefa-C2-L80-b1.dat")
        open(2, file="saidas/tarefa-3/saida-tarefa-C2-L80-b2.dat")
        open(3, file="saidas/tarefa-3/saida-tarefa-C2-L80-b3.dat")
        open(4, file="saidas/tarefa-3/saida-tarefa-C2-L80-b4.dat")
        open(5, file="saidas/tarefa-3/saida-tarefa-C2-L80-b5.dat")

        do i = 1, 5
            call tarefaC2(80, betas(i), i)
            close(1)
        end do

        open(1, file="saidas/tarefa-3/saida-tarefa-C2-L100-b1.dat")
        open(2, file="saidas/tarefa-3/saida-tarefa-C2-L100-b2.dat")
        open(3, file="saidas/tarefa-3/saida-tarefa-C2-L100-b3.dat")
        open(4, file="saidas/tarefa-3/saida-tarefa-C2-L100-b4.dat")
        open(5, file="saidas/tarefa-3/saida-tarefa-C2-L100-b5.dat")

        do i = 1, 5
            call tarefaC2(100, betas(i), i)
            close(1)
        end do
        end
        subroutine tarefaC2(L_real, beta, fname)
!               Tarefa B - Recozimento e quenching
            implicit integer(f-f)
            implicit real(j-j, m-m)
            parameter(L = 100)
            dimension exps(-4:4)
            byte lattice(1:L, 1:L)
            ! periodic boundary conditions
            dimension ipbc(0:L+1)

            do i = 1, L_real
                ipbc(i) = i
            end do  

            ipbc(0) = L_real
            ipbc(L_real+1) = 1

            N = L_real * L_real

            mag = 0.0d0

            call srand(L_real * 392)

            ! half ordered / half random.
            call initialize_lattice(lattice, L_real, L_real)
            call initialize_random_lattice(lattice,  L_real/2, L_real)

            open(99, file = "saidas/tarefa-3/saida-tarefa-C2-conf.dat")
            call write_lattice(lattice, L_real, 99)
            close(99)

            call total_magnetization(lattice, mag, L_real)

            ! initial energy
            E = H_0(lattice, ipbc, L_real)
            dbeta = 0.01
            write(fname, *) 0, E/N
            do i = 1, 3000
                call define_exponentials(exps, beta)
                do k = 1 , N
                    call flip_spin(lattice,ipbc,exps,E,mag,L_real)
                end do   
                write(fname, *) i, E/N
            end do
        end subroutine tarefaC2
\end{minted}


\begin{marginfigure}
    \centering
    \includegraphics[width=\linewidth]{graficos/tarefa-3/graf-tarefa-C2-L80.png}
    \caption{Dinâmica para L=80.}
    \label{fig:c2_l80}
\end{marginfigure}

Partimos dos resultados do item anterior e tentamos obter a temperatura crítica do modelo. Para isso observamos
a variação de energia no intervalo $\beta$ discutido antes, isto é, $0,4 < \beta < 0.6$. 
A imagem (\ref{fig:c2_conf_inicial}) mostra a configuração inicial do sistema. Foram escolhidos alguns 
valores de $\beta$ para executar a dinâmica de Monte Carlo. 


\begin{marginfigure}
    \centering
    \includegraphics[width=\linewidth]{graficos/tarefa-3/graf-tarefa-C2-L60.png}
    \caption{Dinâmica para L=60.}
    \label{fig:c2_l60}
\end{marginfigure}



Nas figuras (\ref{fig:c2_l60}), (\ref{fig:c2_l80}) e (\ref{fig:c2_l100}) estão 
as evoluções, em um intervalo de passos de Monte Carlo reduzido, da energia média por spin. 

\clearpage
Nota-se que as energias médias por spin sempre partem do mesmo valor no intervalo da histerese e a 
que possui maior variação é a que corresponde à $\beta = 0.44$, esse é o $\beta$ relacionado à temperatura crítica $T_c = 1/\beta_c \approx 2,27$ .

\begin{figure}
    \centering
    \includegraphics[width=0.5\linewidth]{graficos/tarefa-3/graf-tarefa-C2-L100.png}
    \caption{Dinâmica para L=100.}
    \label{fig:c2_l100}
\end{figure}

Além disso, pela (\ref{eq:calor_especifico}) podemos constatar que esse $\beta_c = 0,44$ também está associado à 
um valor específico crítico no modelo. 
\section{Tarefa D}

\begin{equation}
    K_B T = \langle \frac{m}{2} \left( v_x^2 + v_y^2 \right) \rangle
\end{equation}
\section{Tarefa E}
\label{sec:secE}
Agora que estudamos a dinâmica molecular e conseguimos atingir observar o que as velocidades 
seguem distribuições de Maxwell-Boltzman, queremos impor uma dinâmica específica sobre o 
gás 2D. 
Nesse caso queremos observar a cristalização de moleculas, e para isso consideramos uma caixa 
de tamanho $L=4$ com $N = 16$ partículas, ou seja, a densidade nesse caso é $\sigma = 1$, o passo 
foi $\Delta t = 0.005$ e a velocidade inicial de teste $v_0 = 1$. 
No entanto, para essa velocidade não foi possível observar a cristalização acontecendo nos diferentes regimes
de temo desejados, por isso foi necessário diminuir a velocidade inicial $v_0$ para 
$v_0 = 0.2$. Isso condiz com um sistema à baixa temperaturas de alta densidade. 

Como pode ver abaixo (\ref{fig:posicoes-finais-e}) correspondem à cristalização das moléculas. 
Da esquerda para a direita temos o ``rastro'' das partículas nos temos $0 \leq t \leq 0.1$, $0.2 \leq t \leq 0.4$ e 
por fim $13 \leq t \leq 16$. Foram consideradas as posições em intervalos de $ 10 \Delta t$. 

\begin{figure}[h!]
    \centering
    \includegraphics[width=0.95\linewidth]{tarefa-E/posicoes-finais.png}
    \caption{Etapas da cristalização das moléculas.}
    \label{fig:posicoes-finais-e}
\end{figure}


Além desse gráfico foi construido um vídeo para essa tarefa. Esse é um pouco mais longo que os anteriores\footnote{E espero que consiga subir ele para o basalto.}
mas observando apenas os segundos finais dele podemos notar que as partículas permanecem quase presas em uma determinada região nas proximidades da 
do ponto de cristalização. É possível até perceber a estrutura triangular com a qual as partículas se movimentam em torno dos vizinhos. 

\subsection*{Implementação - Simulação E}
\input{codigos/tarefa-e.tex}
\clearpage 
\section{Tarefa F}
\begin{figure}[h!]
    \centering
    \includegraphics[width=0.95\linewidth]{tarefa-F/posicoes-finais.png}
    \caption{}
    \label{fig:posicoes-finais-f}
\end{figure}

\begin{figure}[h!]
    \centering
    \includegraphics[width=0.6\linewidth]{tarefa-F/energia.png}
    \caption{Energia total a cada tempo.}
    \label{fig:energia-f}
\end{figure}

\subsection*{Código}
\input{codigos/tarefa-f.tex}


\emph{PS}: O arquivo \verb|tests.f| é um compilado de todas simulações desenvolvidas nesse projeto 
e pode ser compilado para rodar todas elas de uma vez e o script \verb|plots.py| cria todos 
os gráficos e gifs desse trabalho.


\nocite{*}
\printbibliography[title = Referências]
\end{document}
