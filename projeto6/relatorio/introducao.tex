Potencial de Lennard-Jonnes

\begin{equation}
    \mathcal{U}(\vec{r}) = 4 \epsilon \left[ \left(\frac{\sigma}{r} \right)^12 - \left(\frac{\sigma}{r}\right)^6 \right]
    \label{eq:potencial_lanner_jones}
\end{equation}

Utilizando unidades genericas tal que $m=1$, então, pela segunda lei de Newton temos que as componentes 
da velocidade são  

\begin{align}
    a^{x}_{j} = \sum_{j = 1}^{N} \sum_{\ell \neq j}^{N} F_{j, \ell} \cos(\theta_{j, \ell})
    a^{y}_{j} = \sum_{j = 1}^{N} \sum_{\ell \neq j}^{N} F_{j, \ell} \sin(\theta_{j, \ell})
\end{align}

porém é mais interessante encontrar uma relação que não envolva as funções trigonométricas. De fato, 
podemos escrever os senos e cossenos de $\theta_{j, \ell}$ 


$$ \sin(\theta_{j, \ell}) = \frac{dy_{j, \ell}}{d_{j, \ell}} $$
$$ \cos(\theta_{j, \ell}) = \frac{dx_{j, \ell}}{d_{j, \ell}} $$




\subsubsection{Algoritmo de Verlet}

\begin{align}
    t &= n \Delta t \\ 
    x_i(n+1) &= 2 x_i (n) - x_i (n-1)+ a_i^x (n) (\Delta t)^2, n \ge 1 \\ 
    y_i(n+1) &= 2 y_i (n) - y_i (n-1)+ a_i^y (n) (\Delta t)^2, n \ge 1 \\  
    x_i(1) &= x(0) + v_i^x (0) \Delta t \\ 
    y_i(1) &= y(0) + v_i^y (0) \Delta t \\ 
    v_i^x (n) &= \left( x_i(n+1) - x_i(n-1) \right) /(2 \Delta t)\\
    v_i^y (n) &= \left(  y_i(n+1) - y_i(n-1) \right) /(2 \Delta t)
\end{align}
\subsection{Detalhes de implementação}

Módulo para simulações de dinâmica molecular
\begin{minted}{fortran}
    ! Submodules for molecular dynamic simulations
    
    ! Velocity delta 
    function delta_pbc(r_next, r_prev,L)
          implicit real*8(a-h, o-y)
          delta_pbc = r_next - r_prev
          delta_pbc = delta_pbc - L * nint(delta_pbc / L)
    end function delta_pbc

    subroutine initialize_particles(N, L, r_curr,r_prev, v, v0)
          implicit real*8(a-h, o-y)
          dimension r_prev(20, 2)
          dimension r_curr(20, 2)
          dimension v(20, 2)
         
          ! Defining # rows/columns 
          n_cols = ceiling(sqrt(N*1d0))
          n_rows = ceiling((N*1d0)/(n_cols*1d0)) 
          
          ! Spacing 1/4 
          x_spacing = L/(1d0*n_cols)
          y_spacing = L/(1d0*n_rows)
          spacing = min(x_spacing, y_spacing)/4.0 
          
          ! Centering in the grid
          x_offset = x_spacing / 2.0 
          y_offset = y_spacing / 2.0
          call srand(562369)

          k = 1 
          do j = 1, n_rows 
                do i = 1, n_cols 
                      r_curr(k, 1) = (i-1)*x_spacing+x_offset
                      r_curr(k, 2) = (j-1)*y_spacing+y_offset
                      
                      r_curr(k, 1) = r_curr(k,1)+(rand())*spacing
                      r_curr(k, 2) = r_curr(k,2)+(rand())*spacing

                      theta = 2*pi*rand()
                      v(k, 1) = v0*cos(theta)
                      v(k, 2) = v0*sin(theta)
                      
                      r_prev(k, 1) = r_curr(k, 1) - v(k, 1) * dt 
                      r_prev(k, 2) = r_curr(k, 2) - v(k, 2) * dt 
                      k=k+1
                end do 
          end do
    end subroutine initialize_particles

    ! Updates acceleration a = ax, ay 
    ! between particle i and all others
    subroutine compute_acc(N,i,j,L,r_curr,acc, r)
          implicit real*8(a-h, o-y)
          dimension r_curr(20, 2)
          dimension acc(2)
          dimension r(20, 20)
          epsilon = 1e-3

          dx = r_curr(i, 1) - r_curr(j, 1)
          dy = r_curr(i, 2) - r_curr(j, 2)

          dx = dx - L * nint(dx / L)
          dy = dy - L * nint(dy / L)

          r_ij = sqrt(dx**2 + dy**2)
          
          r(i, j) = r_ij 
          r(j, i) = r_ij

          if(r_ij > epsilon .and. r_ij <= 3d0) then 
                F = 24.0 * (2d0/r_ij**13 - 1d0/r_ij**7)
                acc(1) = acc(1) + F * dx / r_ij 
                acc(2) = acc(2) + F * dy / r_ij
          end if 
    end subroutine compute_acc

    subroutine compute_energy(N, L, v, r_curr, E, r)
          implicit real*8(a-h, o-y)
          dimension v(20, 2)
          dimension r_curr(20, 2)
          dimension r(20, 20)
          
          epsilon = 1e-3
          Tk = 0d0
          do i = 1, N
              Tk = Tk + 0.5 * (v(i, 1)**2 + v(i, 2)**2)
          end do
          U = 0d0
          do i = 1, N
            do j = i + 1, N
                r_ij = r(i, j)

                if (r_ij > epsilon .and. r_ij <= 3d0) then
                    U = U + 4 * (r_ij**(-12) - r_ij**(-6))
                end if
            end do
          end do
          E = Tk + U
    end subroutine compute_energy
\end{minted}