A primeira simulação feita foi simplesmente executar a dinâmica do por um 
determinado número de iterações. Iniciamos posicionando as moléculas no espaço 
2D nos centros de malha regular com espaçamentos de $L/\sqrt{N}$ onde $L$ é o tamanho da 
caixa e $N$ o número de partículas e então adicionando um pequeno deslocamento aleatório
em cada uma delas. 

A figura(\ref{fig:posicoes-iniciais-a}) abaixo mostra esse posicionamento inicial definido para os corpos.

\begin{figure}[h!]
    \centering
    \includegraphics[width=0.3\linewidth]{tarefa-A/posicoes-iniciais.png}
    \caption{Posições iniciais das partículas.}
    \label{fig:posicoes-iniciais-a}
\end{figure}

a figura abaixo mostra o  ``rastro'' que as partículas fazem para uma simulação de $500$ iterações. 
Os dados considerados são somentes os iterações multiplas de $3 \Delta t$. Além dessa figura 
também foi gerado uma animação em \verb|gif| da dinâmica molecular, é o arquivo 
localizado na pasta \verb|gráficos/tarefa-A/| entitulado \verb|evolucao.gif|.  

\begin{figure}[h!]
    \centering 
    \includegraphics[width=0.3\linewidth]{tarefa-A/posicoes-finais.png}
    \label{fig:posicoes-finais-a}
    \caption{Coordenadas das partículas projetadas à cada $3 \Delta t$.}
\end{figure}


Também foi construido também um gráfico da energia(\ref{fig:energia-a}) total do sistema para essa simulação.
podemos notar que há variações grandes na energia, mas ela ainda se mantem oscilando em torno de um valor, isso era o esperado. 
Essa variação pode ser devido ao uso de condições periodicas de contorno utilizadas, já que o numéro de partículas e densidade de partículas 
é relativamente pequeno.

\begin{figure}[h!]
    \centering 
    \includegraphics[width=0.3\linewidth]{tarefa-A/energia.png}
    \caption{Energia do sistema à cada $3 \Delta t$.}
    \label{fig:energia_a}
\end{figure}

\clearpage
\subsection{Implementação - Simulação A}
\label{ssec:codigoA}
O código abaixo está no diretório \verb|tarefa-a/| e contém 
as simulações referentes às tarefas A, B e parte da D.
\input{codigos/tarefa-a.tex}
\clearpage